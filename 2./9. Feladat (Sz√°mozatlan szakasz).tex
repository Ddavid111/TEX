\documentclass[12pt,twoside]{article}
\def\MakeUppercaseUnsupportedInPdfStrings{\scshape}
\usepackage{hulipsum}
\usepackage{blindtext}
\usepackage[magyar]{babel}
\usepackage{t1enc}
\usepackage{geometry}
\usepackage{fancyhdr}
\usepackage{xcolor}
\usepackage[colorlinks,linkcolor=blue, urlcolor=purple,citecolor=red]{hyperref}
\usepackage{hyperref}
\hypersetup{linktoc=all}





\geometry{left=5cm,right=5cm,top=3cm,bottom=3cm}
\geometry{bindingoffset=1cm}
\geometry{marginparwidth=3cm}
\geometry{marginparsep=0.5cm}
\geometry{columnsep=2cm}



\begin{document}

\pagestyle{fancy}
\fancypagestyle{plain}{\renewcommand{\headrulewidth}{0pt}\fancyhf{}\fancyfoot[LE,RO]{\thepage}}
\setlength{\headheight}{52pt}


\renewcommand{\footrulewidth}{0.4pt}
\fancyhead[LE,RO]{\thepage}
\fancyfoot[C]{Miskolci Egyetem}
\fancyhead[LO]{\leftmark}
\fancyhead[RE]{\rightmark}












\title{2.óra}
\author{Drig Dávid}
\date{\today}
\renewcommand{\thefootnote}{\fnsymbol{footnote}}
\maketitle



\begin{abstract}
\hulipsum[1]
\blindtext[1]
\footnote{szöveg}
\end{abstract}


\setcounter{tocdepth}{5} 
\tableofcontents
\pagenumbering{roman}

\clearpage

\setcounter{secnumdepth}{5}

\section{Első section}\footnote{szöveg}
\pagenumbering{arabic}
\autoref{s:elso}

\ref{s:masodik}

\ref{s:harmadik}

\autopageref{s:elso}

\pageref{s:masodik}

\pageref{s:harmadik}

\ref{s:szamozatlan}

\pageref{s:szamozatlan}

\subsubsection{subsub}
\subsection{Subsub}
\hulipsum
\phantomsection\label{s:elso}
\subsection{Subsubsub}
\hulipsum

\section[Rövid név]{Hosszú név}
\marginpar{megjegy}
\subsection{Subsection}
\label{s:masodik}
\subsubsection{Subsubsection}
\paragraph{Paragraph}
\section*{Számozatlan section}
\label{s:szamozatlan}
\subparagraph{Subparagraph}

\appendix
\section{Első Section}
\subsection{Első Subsection}
\quote{\hulipsum[2]}
\label{s:harmadik}
\section{Második Section}
\marginpar{ez egy megjegyzés}
\subsection{Második Subsection}
\quotation{\blindtext[2]}
\clearpage
\begin{verse}
Talpra, magyar, hí a haza!
Itt az idő, most vagy soha!
Rabok legyünk, vagy szabadok?
Ez a kérdés, válasszatok! -
A magyarok istenére
Esküszünk,
Esküszünk, hogy rabok tovább
Nem leszünk!

Rabok voltunk mostanáig,
Kárhozottak ősapáink,
Kik szabadon éltek-haltak,
Szolgaföldben nem nyughatnak.
A magyarok istenére
Esküszünk,
Esküszünk, hogy rabok tovább
Nem leszünk!


\end{verse}
\subsection{Második subsection}
\href{https://www.uni-miskolc.hu/~viktoria.vadon/}{hivatkozás link}

\end{document}