\documentclass{article}
\usepackage{hulipsum}
\usepackage{blindtext}
\usepackage[magyar]{babel}
\usepackage{t1enc}

\begin{document}
\title{2.óra}
\author{Drig Dávid}
\date{\today}
\renewcommand{\thefootnote}{\fnsymbol{footnote}}
\maketitle

\begin{abstract}
\hulipsum[1]
\blindtext[1]
\footnote{szöveg}
\end{abstract}


\setcounter{tocdepth}{5} 
\tableofcontents
\pagenumbering{roman}

\clearpage

\setcounter{secnumdepth}{5}

\section{Első section}\footnote{szöveg}
\pagenumbering{arabic}
\subsubsection{subsub}

\subsection{Subsub}
\hulipsum
\subsection{Subsubsub}
\hulipsum

\section[Rövid név]{Hosszú név}
\subsection{Subsection}
\subsubsection{Subsubsection}
\paragraph{Paragraph}
\subparagraph{Subparagraph}

\appendix
\section{Első Section}
\subsection{Első Subsection}
\quote{\hulipsum[2]}
\section{Második Section}
\subsection{Második Subsection}
\quotation{\blindtext[2]}
\clearpage
\begin{verse}
Talpra, magyar, hí a haza!
Itt az idő, most vagy soha!
Rabok legyünk, vagy szabadok?
Ez a kérdés, válasszatok! -
A magyarok istenére
Esküszünk,
Esküszünk, hogy rabok tovább
Nem leszünk!

Rabok voltunk mostanáig,
Kárhozottak ősapáink,
Kik szabadon éltek-haltak,
Szolgaföldben nem nyughatnak.
A magyarok istenére
Esküszünk,
Esküszünk, hogy rabok tovább
Nem leszünk!


\end{verse}

\end{document}