\documentclass{article}
\usepackage[magyar]{babel}
\usepackage{t1enc}
\usepackage{lipsum}
\usepackage{hulipsum}
\usepackage{amsthm}
\usepackage{float}

\newtheorem{tet}{Tétel}
\theoremstyle{definition}
\newtheorem{defin}{Definíció}
\theoremstyle{plain}
\newtheorem{lemma}[tet]{Lemma}
\theoremstyle{remark}
\newtheorem{feladat}{Feladat}[section]
\newtheorem*{megj}{Megjegyzés}
\newfloat{forraskod}{hbt}{lop}
\floatname{forraskod}{Verbatim}

\begin{document}
\listof{forraskod}{Verbatim lista}

\verb|\texttt{verbatim}| szöveg

\verb|\textbf{valami}| szöveg

\verb|\textit{még valami}| szöveg

\hulipsum[2]

\begin{forraskod}
\begin{verbatim}
\begin{tet}[Pitagorasz]
Pitagorasz tétele...
\end{tet}
\end{verbatim}
\caption{Tételkörnyezet}
\end{forraskod}

\hulipsum[2]
\begin{forraskod}
\begin{verbatim}
\begin{enumerate}
\item egy
\begin{itemize}
\item második szint!
\end{itemize}
\item kettő
\end{enumerate}
\end{verbatim}
\caption{Lista}
\end{forraskod}

\hulipsum[2]

\end{document}