\documentclass[twocolumn]{report}
\usepackage[magyar]{babel}
\usepackage{t1enc}
\frenchspacing
\usepackage{hulipsum}
\usepackage{fancyhdr}

\fancyhead[R]{\thepage}
\fancyhead[L]{\nouppercase{\leftmark}}
\fancyfoot[C]{X csoport}

\fancypagestyle{plain}{\fancyhf{} 
\fancyhead[R]{\thepage}}

\usepackage{array} 
\usepackage[table]{xcolor} 
\usepackage{multirow} 

\usepackage{amsmath} 
\usepackage{mathtools} 
\usepackage{amsfonts} 
\usepackage{amsthm}
\theoremstyle{definition} 
\newtheorem{tet}{Tétel}
\newtheorem{defin}{Definíció}

\usepackage{algpseudocode}
\usepackage{algorithm}
\floatname{algorithm}{Algoritmus} 


%%programozás
%% \centerpair parancs, 2 kötelező argumentummal
\newcommand{\centerpair}[2]{%
\par\noindent% %%egész sor kell nekünk
\begin{minipage}[t]{0.48\linewidth}% %%bal oldali doboz, fél* oldalszélesség, függőlegesen felülre igazítva
%%kevesebb, mint fél, hogy középen ne folyjon össze a szöveg!
\begin{flushright}% %%jobbra zárás
#1% %%argumentum beillesztése
\end{flushright}%
\end{minipage}%
\hfill\vline\hfill% %% középvonal és az extra hely kitöltése körülötte középen -- enélkül is teljesítettük a feladatot
\begin{minipage}[t]{0.48\linewidth}%
\begin{flushleft}% %%balra zárás -- fölösleges, mert alapértelmezett
#2%
\end{flushleft}%
\end{minipage}%
\par}

\author{Drig Dávid EZ3YRC}
\title{Zárthelyi dolgozat \\ \Large X csoport}
\date{\today}




\begin{document}

\pagestyle{fancy}


\maketitle


\chapter{Oldalbeállítás}

\section{Első section}
\hulipsum[1-12]

\section{Második section}
\hulipsum[13-24]


\chapter{Táblázat}

\hulipsum
\begin{table*}
\begin{center}
\caption{Egy vállalati kimutatás}
\vspace{2em}
\begin{tabular}{c|c|>{\columncolor{green!20}}r>{\columncolor{red!20}}r>{\columncolor{yellow!20}}r}

év & ágazat & bevétel & kiadás & nettó bevétel\\ \hline
\multirow{4}{2em}{2020} & marketing & xx & xx & xx\\
 & humán erőforrás & \multicolumn{3}{c}{N/A}\\
 & logisztika & xx & xx & xx\\
 & gyártás & xx & xx & xx\\ \hline
\multirow{4}{2em}{2021} & marketing & \cellcolor{white}N/A & xx & \cellcolor{white}N/A\\
 & humán erőforrás & xx & xx & xx\\
 & logisztika & xx & xx & xx\\
 & gyártás & xx & \multicolumn{2}{c}{N/A}\\ 


\end{tabular}
\end{center}
\end{table*}
\hulipsum

\chapter{Matematika}

\begin{defin}[Mátrix szorzás]
Legyenek $m,\textcolor{red}{n},r \in \mathbb{Z}^+, A \in \mathbb{R}^{m \times {\textcolor{red}{n}}}$, és $B \in \mathbb{R}^{{\textcolor{red}{n}} \times  r}$.
Bontsuk fel $A$-t sorvektoraival, $B$-t pedig oszlopvektoraival:
\end{defin}

\begin{equation}
A=\begin{pmatrix}
 & a_1^T & \\
\cellcolor{red!30} & \cellcolor{red!30}{a_2^T} & \cellcolor{red!30}\\
 & \vdots & \\
 & a_m^T &
\end{pmatrix}\text{,}
\quad
B=\begin{pmatrix}
 & \cellcolor{blue!30}\\
 b_1 & {\cellcolor{blue!30}{b_2}} & \cdots & b_r\\
 & \cellcolor{blue!30}\\
 \end{pmatrix}\text{.}
\end{equation}

Mind $a_i$, mind $b_j \in \mathbb{R}^{\textcolor{red}{n}}$, így az $a_i^Tb_j$ skalár szorzat létezik. Az $A \cdot B$ mátrix szorzatot a következőképpen definiáljuk:

\begin{equation}
A \cdot B :=
\begin{pmatrix}
a_1^Tb_1 & \cellcolor{blue!20}{a_1^Tb_2} & \cdots & a_1^Tb_r\\
\rowcolor{red!20}
a_2^Tb_1 & \cellcolor{blue!20}{a_2^Tb_2} & \cdots & a_2^Tb_r\\
\vdots & \cellcolor{blue!20}{\vdots} & \ddots & \vdots\\
a_m^Tb_1 & \cellcolor{blue!20}{a_m^Tb_2} & \cdots & a_m^Tb_r\\
\end{pmatrix} \in \mathbb{R}^{m \times r}
\end{equation}

\begin{tet}[Mátrix szorzás tulajdonságai]
A mátrix szorzás asszociatív, de nem kommutatív művelet.
\end{tet}

\begin{tet}[Mátrix szorzat determinánsa]
Ha $A, b \in \mathbb{R}^{n \times n} $ négyzetes mátrixok, akkor
\end{tet}

\begin{equation}
\det(A \cdot B)=\det(A) \cdot \det(B)\text{.}
\end{equation}

\chapter{Pszeudokód}

\hulipsum
\begin{algorithm}
\caption{Lineáris keresés}
\begin{algorithmic}[3]
\Procedure {linsearch}{A,ertek,@index}
\Require A tömb, ertek a keresett érték
\Ensure index (az első) A-beli index, hogy $A_i$ =ertek, vagy érvénytelen (0)
\State i $\gets$ 1
\While{$A_i \neq$ ertek \textbf{and} i < Hossz[A]}
\State \Call{inc}i
\EndWhile
\If{i < Hossz[A]}
\State index $\gets$ i
\Else
\State index $\gets$ 0
\EndIf
\State \Return index
\EndProcedure
\end{algorithmic}
\end{algorithm}
\hulipsum

\chapter{Programozás}

szöveg előtte... nem teszünk új bekezdést, hogy lássuk, a parancs gondoskodik róla!
\centerpair{bal}{jobb}
szintén új bekezdés nélkül írunk, de automatikusan új bekezdésben (kéne) lennünk

\centerpair{na még egyszer használjuk...}{teszteljük a parancsunkat egy kicsivel hosszabb szövegekkel is\par ez egy új bekezdés. láthatjuk a tördelést és a felülre igazítást is!}


\end{document}