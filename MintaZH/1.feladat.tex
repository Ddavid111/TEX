\documentclass[twocolumn]{report}
\usepackage[magyar]{babel}
\usepackage{t1enc}
\usepackage{hulipsum}
\usepackage{fancyhdr}
\usepackage{geometry}
\usepackage{multirow}
\usepackage{xcolor}
\usepackage{colortbl}
\usepackage{algpseudocode} %%kód írása
\usepackage{algorithm} %%úsztatás
\floatname{algorithm}{Algoritmus} %%átnevezés, magyarítás
\usepackage{amsmath} %%mindig jó ötlet
\usepackage{mathtools} %%pmatrix-hoz
\usepackage{amsfonts} %%\mathbb-hez
\usepackage{amsthm} %%tételekhez
\theoremstyle{definition} %%stílus: vastag betűs cím,  \normalfont törzs. az ez után kiadott \newtheorem parancsokra érvényes, amíg nem írjuk felül
\newtheorem{tet}{Tétel}
\newtheorem{defin}{Definíció} %%def-nek rövidebb lenne, de nem szerencsés, mert \def belső kulcsszó
%% nem varázsoltunk számlálókkal, mert a példában is egymástól és chapter-től is függetlenül számozódtak




\begin{document}



\title{Zárthelyi dolgozat\\\Large X csoport}
\author{Drig Dávid EZ3YRC}
\date{\today}
\maketitle


\fancyhead[R]{\thepage}
\fancyhead[L]{\nouppercase{\leftmark}}
\fancyfoot[C]{X csoport}
\pagestyle{fancy}

\fancypagestyle{plain}{\fancyhead[R]{\thepage}}

\chapter{Első Fejezet}
\section{Első Section}
\hulipsum
\section{Második Section}
\hulipsum

\chapter{Második Fejezet}
\hulipsum
\begin{table*}
\caption{Egy vállalati kimutatás}
\vspace{2em}
\begin{center}

\begin{tabular}{c|c|>{\columncolor{green!20}}r>{\columncolor{red!20}}r>{\columncolor{yellow!20}}r}
év & ágazat & bevétel & kiadás & nettó bevétel\\ \hline
\multirow{4}{2em}{2020} & marketing & xx & xx & xx\\
 & humán erőforrás & \multicolumn{3}{c}{N/A}\\
 & logisztika & xx & xx & xx\\
 & gyártás & xx & xx & xx\\ \hline
\multirow{4}{2em}{2021} & marketing & \cellcolor{white}N/A & xx & \cellcolor{white}N/A\\ 
 & humán erőforrás & xx & xx & xx\\
 & logisztika & xx & xx & xx\\
 & gyártás & xx & \multicolumn{2}{c}{N/A}\\ 
 

\end{tabular}
\end{center}
\end{table*}
\hulipsum

\chapter{Három}

\section{3. fejezet}
\section{Matematika}
\begin{enumerate}
  \item Definíció (Mátrix szorzás). Legyenek $m, n, r \in$ $\mathbb{Z}^{+}, A \in \mathbb{R}^{m \times n}$, és $B \in \mathbb{R}^{n \times r}$. Bontsuk fel $A$-t sorvektoraival, B-t pedig oszlopevektoraival:
\end{enumerate}
$$
A=\left(\begin{array}{c}
a_{1}^{T} \\
a_{2}^{T} \\
\vdots \\
a_{m}^{T}
\end{array}\right), \quad B=\left(\begin{array}{llll}
b_{1} & b_{2} & \cdots & b_{r}
\end{array}\right)
$$
Ha nem csalunk láthatatlan dolgokkal, a következöképp nézne ki a két mátrix - szintén helyes megoldás, csak vizuálisan vektornak látszanak.
$$
A=\left(\begin{array}{c}
a_{1}^{T} \\
a_{2}^{T} \\
\vdots \\
a_{m}^{T}
\end{array}\right), \quad B=\left(\begin{array}{llll}
b_{1} & b_{2} & \cdots & b_{r}
\end{array}\right) .
$$
Mind $a_{i}$, mind $b_{j} \in \mathbb{R}^{n}$, így az $a_{i}^{T} b_{j}$ skalár szorzat létezik. Az $A \cdot B$ mátrix szorzatot a következóképpen definiáljuk:
$$
A \cdot B:=\left(\begin{array}{cccc}
a_{1}^{T} b_{1} & a_{1}^{T} b_{2} & \cdots & a_{1}^{T} b_{r} \\
a_{2}^{T} b_{1} & a_{2}^{T} b_{2} & \cdots & a_{2}^{T} b_{r} \\
\vdots & \vdots & \ddots & \vdots \\
a_{m}^{T} b_{1} & a_{m}^{T} b_{2} & \cdots & a_{m}^{T} b_{r}
\end{array}\right) \in \mathbb{R}^{m \times r}
$$

\begin{enumerate}
  \item Tétel (Mátrix szorzás tulajdonságai). A mátrix szorzás asszociatív, de nem kommutatív múvelet.

  \item Tétel (Mátrix szorzat determinánsa). Ha $A, b \in$ $\mathbb{R}^{n \times n}$ négyzetes mátrixok, akkor

\end{enumerate}
$$
\operatorname{det}(A \cdot B)=\operatorname{det}(A) \cdot \operatorname{det}(B) .
$$

\chapter{Négy}
\hulipsum

\begin{algorithm}
\caption{Lineáris keresés}
\begin{algorithmic}[3]
\Procedure {linsearch}{A,ertek,@index}
\Require A tömb, ertek a keresett érték
\Ensure index (az első) A-beli index, hogy $A_i$=ertek, vagy érvénytelen (0)
\State i $\gets$ 1
\While{ $A_i ne_q$ ertek \textbf{and} i < Hossz[A]}
\State \Call{inc}i
\EndWhile
\If{i<Hossz[A]}
\State index $\gets$ i
\Else
\State index $\gets$ 0
\EndIf
\State \Return index
\EndProcedure
\end{algorithmic}
\end{algorithm}

\hulipsum


\end{document}