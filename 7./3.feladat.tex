\documentclass{article}
\usepackage[magyar]{babel}
\usepackage{t1enc}
\usepackage{lipsum}
\usepackage{hulipsum}
\usepackage{amsmath}
\usepackage{amsfonts}
\usepackage{amssymb}
\usepackage{mathtools}

\newenvironment{vonalzott}%
{\vspace{1ex}\hrule\vspace{1ex}}%
{\vspace{1ex}\hrule\vspace{1ex}}

\newenvironment{vonalzott*}[1][Kulcsgondolatok]{\begin{center}\begin{minipage}{0,8\textwidth}%
\vspace{1ex}\hrule\vspace{1ex}\begin{center}\Large\textbf{#1}\end{center}}{\vspace{1ex}\hrule\vspace{1ex}%
\end{minipage}\end{center}}


\begin{document}

\begin{vonalzott*}
\newcommand{\kgitem}{\par\makebox[1 cm]{\stepcounter{szamlalo} \theszamlalo}}
\newcounter{szamlalo}
\newcounter{osszamlalo}
\counterwithin{szamlalo}{osszamlalo}
\kgitem
\section{Elso}
\stepcounter{osszamlalo}
\kgitem
safddsa
asdf
sadfsa
sdaffas
asfd
adsfas
sdfafd
\kgitem
\kgitem
\kgitem
\kgitem
\section{Masodik}
\stepcounter{osszamlalo}
\kgitem
\hulipsum[2]
\kgitem
\kgitem kdlsjf
\kgitem lkdsa
A számláló értéke: \theszamlalo
\end{vonalzott*}

\end{document}